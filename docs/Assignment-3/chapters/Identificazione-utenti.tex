\documentclass[../Assignment-3-LPSMT.tex]{subfiles}
\graphicspath{{\subfix{../img/}}}

\begin{document}

\chapter{Identificazione del segmento utente}

In Europa, come nel resto del mondo, il mercato dei lettori di manga è
in costante aumento~\cite{mangaOut}.\\
Pur ampliandosi a fasce d'età superiori, il mercato dei Manga rimane
molto legato ad un pubblico giovane che preferisce il formato digitale.\\
Il formato digitale viene preferito per una serie di ragioni:

\begin{itemize}
	\item Costi inferiori rispetto alle coppie fisiche, un volume standard costa almeno \euro{5}
	\item Tempi di attesa di per ricevere volume/capitolo
	\item Reperibilità da un catalogo vastissimo
	\item Facilita un primo approccio ad una nuova serie
\end{itemize}

Per una velocità della lettura e portabilità la maggior parte degli utenti preferisce leggere i propri Manga su un dispositivo mobile~\cite{NLTreport}.\\
Per i motivi sopra citati la fascia d'età che potrebbe goderemaggiormente della nostra applicazione sono i ragazzi/rgazze tra i 12 ed i 18 anni ed i giovani adulti tra i 19 e 25 anni.

\end{document}
