\documentclass[../Assignment-3-LPSMT.tex]{subfiles}
\graphicspath{{\subfix{../img/}}}

\begin{document}

\chapter{Introduzione}

La maggior parte delle applicazioni/piattaforme di lettura si basa su un
modello ad abbonamento, un utente pagando una quota mensile può leggere
tutto il catalogo dei manga della piattaforma.

Il problema di queste applicazioni è che vincola l'utente a dover pagare
un abbonamento per poter leggere manga che potrebbe già possedere in
formato virtuale.

Le piattaforme/applicazioni per la catalogazione delle letture sono per
lo più gratis, ma permettono solo di gestire le proprie letture senza
alcuna forma di reader in app.

La nostra idea è quella di colmare questa fascia di mercato,
\textbf{Manga-check} sarà questo il nome, dovrà permettere all'utente di
leggere qualsiasi manga lui possegga (formato \emph{.cbz}) e tramite una GUI
dare la possibilità di catalogare le proprie letture.

L'utente avrò anche la possibilità di eseguire un log-in per mantenere
la propria lista delle letture su più dispositivi.

L'applicazione darà la possibilità di leggere sia manga che sono nella
memoria del dispositivo che si sta utilizzando, ma anche di leggere da
un server FTP remoto del quale l'utente possiede le credenziali.

\section{Stack tecnologico}
\begin{itemize}
	\item Version control: git
	\item Online repository: github
	\item Editor: Android Studio
	\item Language: Kotlin
	\item Design style: Material
	\item DBMS: SQLite o PostgresSQL
\end{itemize}

\end{document}
