\documentclass[../Assignment-3-LPSMT.tex]{subfiles}
\graphicspath{{\subfix{../img/}}}

\begin{document}

\chapter{Analisi critica dei limiti dell’applicazione}

Per quanto strutturata e testata l'applicazione presenta margini di miglioramento.\\
Primo tra tutti lo stile grafico, che non fornisce un'identità propria e purtroppo non è ben integrabile con il nuovo paradigma di colorazione~\cite{matDesColor} basato sul wallpaper introdotto in Material Design 3.\\
Per questioni di tempo e semplicità non ci è stato possibile implementare il log in dell'utente e abbiamo dovuto optare per una soluzione semplificata anche se comunque efficente.\\
Il sistema di importing della reading list utilizzato, allo stato attuale, è da intendersi solo come un sistema per spostare i dati da un'installazione all'altra, inafatti comporta una sovrascrittura completa della precedente reading list.\\
Neppure il database ed il sistema di query non è perfetto, puo infattti capitare che la descrizione di un qualche manga sia vuota e venga quindi restituita la seguente stringa \verb+...)];+.\\
L'applicazione presenta anche alcuni margini di migloramento:
\begin{itemize}
  \item L'implementazione del pintch to zoom nella sezione di reading del manga.
  \item L'attivazione di un opt in per visualizzare anche fumetti per adulti.
  \item L'ampliamento del database per il supporto anche ai comic occidentali.
\end{itemize}

\end{document}
